\documentclass[12pt]{report}
\usepackage{amsmath}
\usepackage{amssymb}
\usepackage{listings}
\usepackage{hyperref}

\lstset{
language=C,
basicstyle=\small\sffamily,
numbers=left,
numberstyle=\tiny,
frame=tb,
columns=fullflexible,
showstringspaces=false
}

\setlength{\parindent}{0cm}
\setlength{\parskip}{\baselineskip}
\begin{document}

\section*{Project Progress Report}
\author*{Julian Knodt} \\
\today \\
Systems \& ML

Updates on the project are as follows:
I switched from audio compression to working on efficient sparse matrix representations. In
doing so, I decided to look mainly at efficient compression algorithms. Initially, I was curious
how tries might work for efficiently representing matrices.

This lead me to the conception that matrices can be treated like ordered maps, where the list of
indeces is the "string" key and the corresponding output value is the associated value. Finite
state transducers(FST), as opposed to tries, also compress suffixes of keys, so I plan on
implementing a sparse matrix with the underlying structure being a FST. I'm basing my work on a
finite state transducer \href{https://github.com/BurntSushi/fst/tree/master/src}{library} and
then planning on building a wrapper on top that supports the creation of matrices from other
static formats, adding scalars, and vector multiplication to start.

I've currently looked through the existing library, and extended it to support floating point
associated values. I further need to modify the query library to support matrix like indexing,
as well as storing the shape of a matrix. I think now that I have a fundamental understanding of
the library, the rest should not take too long.


\end{document}
